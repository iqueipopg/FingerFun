\chapter{Resultados.}

Las siguientes son las principales funcionalidades y logros alcanzados durante el desarrollo del proyecto:

\section{Funcionalidades Principales}

\subsection{Detección precisa de contraseñas gestuales:}

El sistema permite reconocer contraseñas basadas en gestos mediante comparación con plantillas predefinidas, garantizando un nivel de seguridad óptimo.

La detección incluye retroalimentación inmediata para indicar si la contraseña es correcta o no, habilitando o bloqueando el acceso al sistema según corresponda.

\subsection{Seguimiento de secuencias de formas:}

FingerFun genera secuencias aleatorias de figuras geométricas que los usuarios deben reproducir en orden correcto.

La reproducción exitosa de la secuencia permite al jugador avanzar al siguiente nivel, aumentando la dificultad progresivamente.

\subsection{Sistema de vidas:}

Cada error en la reproducción de las secuencias reduce el número de vidas del usuario.

Al perder todas las vidas, el juego finaliza, proporcionando una experiencia desafiante y entretenida.

\subsection{Gestor de niveles:}

La estructura del juego está dividida en niveles. Cada nivel introduce una nueva combinación de formas, aumentando la complejidad del reto.

\subsection{Visualización en tiempo real:}

La trayectoria de los gestos se muestra en pantalla, permitiendo al usuario ver en tiempo real su interacción con el sistema.

Esta funcionalidad mejora la comprensión del usuario sobre su desempeño y facilita la corrección de errores.

\section{Logros Destacados:}

\textbf{Integración de tecnologías avanzadas:} Utiliza algoritmos de visión por ordenador (MediaPipe y OpenCV) y métodos de predicción (como filtros de Kalman) para ofrecer una experiencia precisa y fluida.

\textbf{Interfaz intuitiva:} Diseñada para ser accesible incluso para usuarios sin experiencia técnica, ofreciendo retroalimentación visual clara.

\textbf{Impacto educativo y de entretenimiento:} Combinando desafíos cognitivos con elementos de juego, FingerFun es una herramienta ideal tanto para el entretenimiento como para el aprendizaje.

\section{Conclusión}

FingerFun ha cumplido con creces los objetivos iniciales del proyecto, destacándose como un algoritmo versátil para el reconocimiento de gestos y validación de contraseñas.
