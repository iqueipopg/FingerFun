\chapter{Futuros Desarrollos.}

El sistema FingerFun posee un gran potencial gracias a su diseño modular y su capacidad para reconocer y procesar gestos de manera precisa y eficiente. A continuación, se enumeran algunas ideas para futuros desarrollos que podrían potenciar sus capacidades:

\section{Asistente de corrección técnica:}

Integrar un módulo de aprendizaje automático entrenado con imágenes de profesionales realizando acciones específicas (como movimientos deportivos o ejercicios de rehabilitación). Este módulo podría analizar la técnica del usuario y proporcionar retroalimentación en tiempo real para corregir errores y mejorar el rendimiento.

\section{Educación infantil interactiva:}

Desarrollar un sistema que permita a los niños aprender el abecedario y practicar la escritura de letras mediante gestos realizados con sus manos. Este enfoque lúdico podría incentivar el aprendizaje al combinar tecnología y juego.

\section{Herramienta de aprendizaje de idiomas:}

Incorporar gestos para enseñar la escritura y pronunciación de caracteres en diferentes idiomas (como chino, japonés o árabe), ofreciendo una experiencia inmersiva para los usuarios.

\section{Aplicaciones en fisioterapia y rehabilitación:}

Implementar un módulo que evalúe la precisión de movimientos durante terapias de rehabilitación física, ayudando tanto a terapeutas como a pacientes a medir su progreso.

